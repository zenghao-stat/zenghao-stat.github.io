% !TEX TS-program = xelatex

%-------------------------
% 中文简历(基于英文版)
% Author : Hao Zeng
%------------------------

\documentclass[letterpaper,10pt]{article}

\usepackage{latexsym}
\usepackage[empty]{fullpage}
\usepackage{titlesec}
\usepackage{marvosym}
\usepackage[usenames,dvipsnames]{color}
\usepackage{verbatim}
\usepackage{enumitem}
\usepackage[hidelinks]{hyperref}
\usepackage{fancyhdr}
\usepackage{tabularx}
\usepackage{etaremune}
% \usepackage{fontawesome} % 移除避免 XeLaTeX 字体问题
% \input{glyphtounicode} % XeLaTeX 不需要,且可能缺失
\usepackage[UTF8]{ctex}
\usepackage{fontawesome5}
\usepackage{lastpage}
\usepackage{ifthen}

\newif\ifzh
\zhtrue

\pagestyle{fancy}
\fancyhf{} % 清空页眉页脚
\fancyfoot{}
\renewcommand{\headrulewidth}{0pt}
\renewcommand{\footrulewidth}{0pt}

% 页脚设置:仅在最后一页右下角显示更新时间
\newcommand{\UpdatedOn}{\number\year-\ifnum\month<10 0\fi\number\month-\ifnum\day<10 0\fi\number\day}
\setlength{\footskip}{12pt}
\fancyfoot[R]{\scriptsize \ifthenelse{\equal{\thepage}{\pageref{LastPage}}}{Updated:\UpdatedOn}{}}

% 页边距调整
\addtolength{\oddsidemargin}{-0.5in}
\addtolength{\evensidemargin}{-0.5in}
\addtolength{\textwidth}{1in}
\addtolength{\topmargin}{-.5in}
\addtolength{\textheight}{1.0in}

\urlstyle{same}

\raggedbottom
\raggedright
\setlength{\tabcolsep}{0in}

% 章节格式
\titleformat{\section}{
  \vspace{-4pt}\scshape\raggedright\large
}{}{0em}{}[\color{black}\titlerule \vspace{-5pt}]

% PDF可机读/ATS可解析(pdfLaTeX 专用;XeLaTeX 不需要)
\ifdefined\pdfgentounicode
  \pdfgentounicode=1
\fi

%-------------------------
% 自定义命令(沿用英文版的结构)
\newcommand{\resumeItem}[1]{
  \item\small{
    {#1 \vspace{-2pt}}
  }
}

\newcommand{\resumeSubheading}[4]{
  \vspace{-2pt}\item
    \begin{tabular*}{0.97\textwidth}[t]{l@{\extracolsep{\fill}}r}
      \textbf{#1} & #2 \\
      \textit{\small#3} & \textit{\small #4} \\
    \end{tabular*}\vspace{-7pt}
}

\newcommand{\resumeSubSubheading}[2]{
    \item
    \begin{tabular*}{0.97\textwidth}{l@{\extracolsep{\fill}}r}
      \textit{\small#1} & \textit{\small #2} \\
    \end{tabular*}\vspace{-7pt}
}

\newcommand{\resumeProjectHeading}[2]{
    \item
    \begin{tabular*}{0.97\textwidth}{l@{\extracolsep{\fill}}r}
      \small#1 & #2 \\
    \end{tabular*}\vspace{-7pt}
}

\newcommand{\resumeSubItem}[1]{\resumeItem{#1}\vspace{-4pt}}

\renewcommand\labelitemii{$\vcenter{\hbox{\tiny$\bullet$}}$}

\newcommand{\resumeSubHeadingListStart}{\begin{itemize}[leftmargin=0.15in, label={}]}
\newcommand{\resumeSubHeadingListEnd}{\end{itemize}}
\newcommand{\resumeItemListStart}{\begin{etaremune}}
\newcommand{\resumeItemListEnd}{\end{etaremune}\vspace{-5pt}}

%-------------------------------------------
%%%%%%  简历从这里开始  %%%%%%%%%%%%%%%%%%%%%%%%%%%%

\begin{document}

\begin{center}
    \textbf{\Huge \scshape 曾浩}
    \vspace{0.5em}\\
    {\small 博士后,南方科技大学统计与数据科学系}
\end{center}

\begin{center}
    \small \faIcon{envelope}\ \href{mailto:zenghao.acmail@gmail.com}{zenghao.acmail@gmail.com} $|$
    \faIcon{graduation-cap}\ \href{https://scholar.google.com/citations?user=-EiBHeIAAAAJ&hl=en}{Google Scholar} $|$
    \faIcon{home}\ \href{https://zenghao-stat.github.io}{个人主页:zenghao-stat.github.io}
\end{center}

%-----------EDUCATION-----------
\ifzh
  \section{教育经历}
  \resumeSubHeadingListStart
    \resumeSubheading
      {邹至庄经济研究院,厦门大学}{2020年9月 -- 2024年6月}
      {统计学,理学博士,导师:钟威, 许杏柏, 刘拓}{}
    
    \resumeSubheading
      {王亚南经济研究院,厦门大学}{2018年9月 -- 2020年6月}
      {数量经济学,经济学硕士,导师:钟威}{转博至统计学博士}
    
    \resumeSubheading
      {数学学院,山东大学}{2014年9月 -- 2018年6月}
      {数学学士(彭实戈班:金融数学与金融工程基地班)}{}
  \resumeSubHeadingListEnd
\else
  \section{Education}
  \resumeSubHeadingListStart
    \resumeSubheading
      {Gregory and Paula Chow Institute for Studies in Economics, Xiamen University}{Sept. 2020 -- Jun. 2024}
      {Ph.D. in Statistics, Supervised by Wei Zhong, Xingbai Xu, and Tuo Liu}{}
    
    \resumeSubheading
      {Wang Yanan Institute for Studies in Economics, Xiamen University}{Sept. 2018 -- Jun. 2020}
      {M.S. in Quantitative Economics}{Transfer to Ph.D. in Statistics}
    
    \resumeSubheading
      {School of Mathematics, Shandong University}{Sept. 2014 -- Jun. 2018}
      {B.S. in Mathematics (Peng's Class: Base of Financial Mathematics \& Financial Engineering)}{}
  \resumeSubHeadingListEnd
\fi

%-----------RESEARCH INTERESTS-----------
\ifzh
  \section{研究领域}
  \ifbusiness
    % 预留 business 内容
  \else
    \begin{itemize}[leftmargin=0.15in, label={}]
        \item{
         \textbf{统计机器学习}{:模型无关统计推断、共型预测、迁移学习} \\
         \textbf{交叉学科研究}{:大型语言模型、空间统计、计量经济学、生物统计}
        }
    \end{itemize}
  \fi
\else
  \section{Research Area}
  \ifbusiness
    \begin{itemize}[leftmargin=0.15in, label={}]
        \small{\item{
        \textbf{Business Intelligence Decision}{: Predictive Analytics, Fairness in AI, and Data-Driven Decision Support Systems}\\
        \textbf{Statistical Machine Learning}{: Model Free Predictive Inference, Conformal Prediction, Transfer Learning} \\
        \textbf{Interdisciplinary Research}{: Large Language Models for Business Intelligence, and Responsible AI in Commercial Applications}
        }}
    \end{itemize}
  \else
    \begin{itemize}[leftmargin=0.15in, label={}]
        \small{\item{
         \textbf{Statistical Machine Learning}{: Model Free Predictive Inference, Conformal Prediction, Transfer Learning} \\
         \textbf{Interdisciplinary Research}{: Large Language Models, Spatial Statistics, Econometrics, and Biostatistics}
        }}
    \end{itemize}
  \fi
\fi

%-----------EXPERIENCE-----------
\ifzh
  \section{工作与教学经历}
  \resumeSubHeadingListStart
    \resumeSubheading
      {南方科技大学、新加坡国立大学联合博士后项目}{2024年7月 -- 至今}
      {统计学博士后,导师:荆炳义,魏鸿鑫,周望}{}
    \resumeSubheading
      {统计与数据科学系, 新加坡国立大学}{2023年5月 -- 2023年10月}
      {访问学生,导师:余涛}{}
    \resumeSubheading
      {经济学院,厦门大学}{2019年9月 -- 2022年6月}
      {助教:高级计量经济学 \textup{I}(研)、高级概率论(研)、实变函数(本)两次、概率论导论(本)}{}
    \resumeSubheading
      {埃森哲(中国)}{2018年4月 -- 2018年8月}
      {电力企业管理咨询项目}{}
  \resumeSubHeadingListEnd
\else
  \section{Experience}
  \resumeSubHeadingListStart
    \resumeSubheading
      {Southern University of Science and Technology \& National University of Singapore}{Jul. 2024 -- Present}
      {Postdoctoral Researcher in Statistics, Supervised by Bingyi Jing, Hongxin Wei, and Wang Zhou}{}
    
    \resumeSubheading
      {Department of Statistics and Data Science, National University of Singapore}{May 2023 -- Oct. 2023}
      {Visiting Researcher, Cooperated with Tao Yu}{}
    
    \resumeSubheading
      {School of Economics, Xiamen University}{Fall 2022}
      {Teaching Assistant in Probability Introduction (Undergraduate)}{}
    
    \resumeSubheading
      {School of Economics, Xiamen University}{Spring 2022}
      {Teaching Assistant in Real Analysis (Undergraduate)}{}
    \resumeSubheading
      {School of Economics, Xiamen University}{Fall 2021}
      {Teaching Assistant in Advanced Probability Theory (Postgraduate)}{}
    
    \resumeSubheading
      {School of Economics, Xiamen University}{Spring 2021}
      {Teaching Assistant in Real Analysis (Undergraduate)}{}
    
    \resumeSubheading
      {School of Economics, Xiamen University}{Fall 2019}
      {Teaching Assistant in Advanced Econometrics I (Postgraduate)}{}
    \resumeSubheading
      {Accenture (China)}{Apr. 2018 -- Aug. 2018}
      {Power Enterprise Management Consulting Project}{}
  \resumeSubHeadingListEnd
\fi

%-----------PUBLICATIONS-----------
\ifzh
  \section{学术成果}
  (* 表示通讯作者,$^\dagger$ 表示共同贡献,$^\diamond$ 表示指导学生。)
  \resumeSubHeadingListStart
  \resumeProjectHeading{\textbf{期刊文章}}{}
      \resumeItemListStart
        \resumeItem{\textbf{Hao Zeng}, Wei Zhong, \& Xingbai Xu (2025). ``Transfer Learning for Spatial Autoregressive Models with Application to U.S. Presidential Election Prediction.'' \textbf{Journal of Business \& Economic Statistics}. (已接收,统计学、计量经济学顶级期刊, ABS 4\(\star\))} 
        \resumeItem{Kangdao Liu, Tianhao Sun, \textbf{Hao Zeng}, Yongshan Zhang, Chi-Man Pun, \& Chi-Man Vong (2025). ``Spatial-Aware Conformal Prediction for Trustworthy Hyperspectral Image Classification.'' \textbf{IEEE Transactions on Circuits and Systems for Video Technology} 35 (9): 8754–8766. (中科院一区,计算机顶级期刊)}
        \resumeItem{\textbf{Hao Zeng}, Chuang Wan, Wei Zhong, \& Tuo Liu (2024). ``Robust Integrative Analysis via Quantile Regression with Homogeneity and Sparsity.'' \textbf{Journal of Statistical Planning and Inference} 234 (January): 106196.}
        \resumeItem{Chuang Wan$^\dagger$, \textbf{Hao Zeng$^\dagger$*}, Wenyang Zhang$^\dagger$, Wei Zhong$^\dagger$, \& Changliang Zou$^\dagger$ (2024). ``Data‐driven Estimation for Multithreshold Accelerated Failure Time Model.'' \textbf{Scandinavian Journal of Statistics} 52 (1): 447-468.}
        \resumeItem{Wenqian Chen, \textbf{Hao Zeng}, Xiaoya Wang, Qiuping Xu, Panpan Liu, Liwei Zhang, Yingyue Hou, Qing Luo, Xueye Liu, Zhe Jiang, Zhiyuan Zhou, Jiang Chen, \& Jing Guo (2022). ``A Structural Equation Modeling Approach to Determine the Correlation between the Vertical and Sagittal Skeletal Patterns and Posterior Basal Bones Mismatching in Patients with Skeletal Class III Malocclusion.'' \textbf{American Journal of Orthodontics and Dentofacial Orthopedics} 162 (6): e277–294. (封面文章)}
        \resumeItem{Wenqian Chen, \textbf{Hao Zeng}, Luna Sun, Qiuping Xu, Zhenxue Chen, Yunhan Sun, Qi Jia, Chengyun Liu, \& Jing Guo (2021). ``Match of the Bimaxillary Basal Bone Arches and Its Variations among Individuals.'' \textbf{Scanning} 2021 (1): 9625893.}
      \resumeItemListEnd
    
    \resumeProjectHeading{\textbf{会议文章}}{}
    \resumeItemListStart
        \resumeItem{\textbf{Hao Zeng$^\dagger$}, Kangdao Liu$^\dagger$, Bingyi Jing, \& Hongxin Wei (2025). ``Parametric Scaling Law of Tuning Bias in Conformal Prediction.'' \textbf{42nd International Conference on Machine Learning (ICML 2025)}. (AI/ML 顶级会议)}
        \resumeItem{Huajun Xi$^\diamond$, Kangdao Liu, \textbf{Hao Zeng}, Wenguang Sun, \& Hongxin Wei (2025). ``Exploring the Noise Robustness of Online Conformal Prediction.'' \textbf{39th Annual Conference on Neural Information Processing Systems (NeurIPS 2025)}. (AI/ML 顶级会议)}
        \resumeItem{Kangdao Liu, \textbf{Hao Zeng}, Jianguo Huang, Huiping Zhuang, Chi-Man Vong, \& Hongxin Wei (2025). ``C-Adapter: Adapting Deep Classifiers for Efficient Conformal Prediction Sets.'' \textbf{28th European Conference on Artificial Intelligence}.}
    \resumeItemListEnd
    
    \resumeProjectHeading
      {\textbf{工作论文}}{}
      \resumeItemListStart
        \resumeItem{张静怡$^\diamond$,岳阳,钟威,曾浩* (2025). ``分组网络向量自回归的稳健估计——以全国空气质量数据分析为例.'' (已提交至 \textbf{统计研究})}
        \resumeItem{\textbf{Hao Zeng}, \& Chao Liu (2025). ``TL-FCSAR: A Transfer Learning Model for Heterogeneous Spatial Effects in U.S. Presidential Elections.'' (已提交至 \textbf{Spatial Statistics})}
        \resumeItem{Wenqian Chen$^\dagger$, \textbf{Hao Zeng}$^\dagger$, Jiawen Lei, Yuanning Hou, Chunchun Du, Yiyao Zhang, Yi Lin, Xinan Chen, \& Tianrong He (2025). ``Sex-specific Associations between Craniofacial Skeletal Patterns and Condylar Morphology and Position in Temporomandibular Joint Osteoarthritis: A Three-dimensional Latent Variable Analysis.'' (已提交至 \textbf{Journal of Dentistry})}\vspace{-4pt}%
        \resumeItem{Huipeng Huang$^\diamond$, Wenbo Liao, Huajun Xi, \textbf{Hao Zeng}, Mengchen Zhao, \& Hongxin Wei (2025). ``Selective Labeling with False Discovery Rate Control.'' (提交至 \textbf{The 14th International Conference on Learning Representations (ICLR 2026)})}
        \resumeItem{Zhenlong Liu$^\diamond$, \textbf{Hao Zeng}, Weiran Huang, \& Hongxin Wei (2025). ``High-Power Training Data Identification with Provable Statistical Guarantees.'' (提交至 \textbf{The 14th International Conference on Learning Representations (ICLR 2026)})}
        \resumeItem{\textbf{Hao Zeng}$^\dagger$, Jianguo Huang$^\dagger$\(^\diamond\), Bingyi Jing, Hongxin Wei, \& Bo An (2025). ``PAC Reasoning: Controlling the Performance Loss for Efficient Reasoning.'' (提交至 \textbf{The 14th International Conference on Learning Representations (ICLR 2026)})}
        \resumeItem{Hongfu Gao, Feipeng Zhang, \textbf{Hao Zeng}, Deyu Meng, Bingyi Jing, \& Hongxin Wei (2025). ``Exploring Imbalanced Annotations for Effective In-Context Learning.''(提交至 \textbf{The 14th International Conference on Learning Representations (ICLR 2026)})}
        \resumeItem{Xuanning Zhou$^\diamond$, Zihao Shi$^\diamond$, \textbf{Hao Zeng}, Xiaobo Xia, Bingyi Jing, \& Hongxin Wei (2025). ``Semi-Supervised Conformal Prediction with Unlabeled Nonconformity Score.'' (已提交至 \textbf{The IEEE/CVF Conference on Computer Vision and Pattern Recognition 2026 (CVPR 2026)})}
        \resumeItem{\textbf{Hao Zeng}, Bingyi Jing, \& Hongxin Wei (2025). ``Conditional Tuning in Conformal Prediction.''}
        \resumeItem{\textbf{Hao Zeng}, Bingyi Jing, \& Hongxin Wei (2025). ``The Double Descent of Conformal Prediction.''}
        \resumeItem{\textbf{Hao Zeng}, Kangdao Liu, Bingyi Jing, \& Hongxin Wei (2025). ``On Tuning Bias in Conformal Prediction.''}
        \resumeItem{\textbf{Hao Zeng}, \& Jianguo Huang$^\diamond$ (2025). ``PAC Accelerate: A Statistically Guaranteed Acceleration Method for Large Generative Models.''}
        \resumeItem{\textbf{Hao Zeng} (2025). ``A Note on the Impossibility of Conditional PAC-Efficient Reasoning.''}
        \resumeItem{\textbf{Hao Zeng}, \& Jianguo Huang$^\diamond$ (2025). ``Group-Conditional PAC Reasoning: Groupwise Efficient Reasoning with Statistical Guarantees.''}
        \resumeItem{\textbf{Hao Zeng}, Jiajun Sun$^\diamond$, Bingyi Jing, \& Wei Zhong (2025). ``Test-time Valid Selection of Conformal Sets via Gaussian Stability.''}
        \resumeItem{\textbf{Hao Zeng}, Junxian Liu$^\diamond$, Bingyi Jing, \& Hongxin Wei (2025). ``Efficient Backward Conformal Prediction.''} 
        \resumeItem{\textbf{Hao Zeng}, Wei Zhong, Chuang Wan (2025). "Selective Inference for the Identified Groups in Network Vector Autoregressive Model."}
        \resumeItem{\textbf{Hao Zeng}, Huipeng Huang$^\diamond$, Jianguo Huang$^\diamond$, Bingyi Jing, \& Hongxin Wei (2025). ``HyPAC:\@ Cost-Efficient LLM--Human Hybrid Annotation with PAC Error Guarantees.''}
        \resumeItem{Xiaoqi Qiu$^\dagger$$^\diamond$, \textbf{Hao Zeng}$^\dagger$, Bingyi Jing, \& Hongxin Wei (2025). ``LLM Provenance Testing via Constrained Model Confidence Sets.''}
      \resumeItemListEnd
    
    \resumeProjectHeading
      {\textbf{开源软件}}{}
      \resumeItemListStart
        \resumeItem{Chuang Wan, \textbf{Hao Zeng}, Wei Zhong, \& Changliang Zou (2023). ``MTAFT: Data-Driven Estimation for Multi-Threshold Accelerate Failure Time Model.'' \href{https://cran.r-project.org/web/packages/MTAFT/index.html}{https://cran.r-project.org/web/packages/MTAFT/index.html}}
      \resumeItemListEnd
  \resumeSubHeadingListEnd
\else
  \section{Publications}
  (* means corresponding author, $^\dagger$ means equal contribution, \(^\diamond\) means supervised students.)
  \resumeSubHeadingListStart
  \resumeProjectHeading{\textbf{Journal Articles}}{}
  \resumeItemListStart
    \resumeItem{\textbf{Hao Zeng}, Wei Zhong, \& Xingbai Xu (2025). ``Transfer Learning for Spatial Autoregressive Models with Application to U.S. Presidential Election Prediction.'' (Accepted for \textbf{Journal of Business \& Economic Statistics}, econometrics and statistics top journal, ABS 4\(\star\))} 
    \resumeItem{Kangdao Liu\(^\diamond\), Tianhao Sun, \textbf{Hao Zeng}, Yongshan Zhang, Chi-Man Pun, \& Chi-Man Vong (2025). ``Spatial-Aware Conformal Prediction for Trustworthy Hyperspectral Image Classification.'' \textbf{IEEE Transactions on Circuits and Systems for Video Technology} 35 (9): 8754–8766. (CAS Q1 journal, Top)}
    \resumeItem{\textbf{Hao Zeng}, Chuang Wan, Wei Zhong, \& Tuo Liu (2024). ``Robust Integrative Analysis via Quantile Regression with Homogeneity and Sparsity.'' \textbf{Journal of Statistical Planning and Inference} 234 (January): 106196.}
    \resumeItem{Chuang Wan$^\dagger$, \textbf{Hao Zeng$^\dagger$*}, Wenyang Zhang$^\dagger$, Wei Zhong$^\dagger$, \& Changliang Zou$^\dagger$ (2024). ``Data‐driven Estimation for Multithreshold Accelerated Failure Time Model.'' \textbf{Scandinavian Journal of Statistics} 52 (1): 447–468.}
    \resumeItem{Wenqian Chen, \textbf{Hao Zeng}, Xiaoya Wang, Qiuping Xu, Panpan Liu, Liwei Zhang, Yingyue Hou, Qing Luo, Xueye Liu, Zhe Jiang, Zhiyuan Zhou, Jiang Chen, \& Jing Guo (2022). ``A Structural Equation Modeling Approach to Determine the Correlation between the Vertical and Sagittal Skeletal Patterns and Posterior Basal Bones Mismatching in Patients with Skeletal Class III Malocclusion.'' \textbf{American Journal of Orthodontics and Dentofacial Orthopedics} 162 (6): e277–294. (Cover paper)}
    \resumeItem{Wenqian Chen, \textbf{Hao Zeng}, Luna Sun, Qiuping Xu, Zhenxue Chen, Yunhan Sun, Qi Jia, Chengyun Liu, \& Jing Guo (2021). ``Match of the Bimaxillary Basal Bone Arches and Its Variations among Individuals.'' \textbf{Scanning} 2021 (1): 9625893.}
  \resumeItemListEnd

    \resumeProjectHeading{\textbf{Conference Papers}}{}
    \resumeItemListStart
      \resumeItem{\textbf{Hao Zeng$^\dagger$}, Kangdao Liu$^\dagger$, Bingyi Jing, \& Hongxin Wei (2025). ``Parametric Scaling Law of Tuning Bias in Conformal Prediction.'' \textbf{42nd International Conference on Machine Learning (ICML 2025)}. (AI/ML Top)}
      \resumeItem{Huajun Xi$^\diamond$, Kangdao Liu, \textbf{Hao Zeng}, Wenguang Sun, \& Hongxin Wei (2025). ``Exploring the Noise Robustness of Online Conformal Prediction.'' \textbf{39th Annual Conference on Neural Information Processing Systems (NeurIPS 2025)}. (AI/ML Top)}
      \resumeItem{Kangdao Liu\(^\diamond\) , \textbf{Hao Zeng}, Jianguo Huang, Huiping Zhuang, Chi-Man Vong, \& Hongxin Wei (2025). ``C-Adapter: Adapting Deep Classifiers for Efficient Conformal Prediction Sets.'' \textbf{28th European Conference on Artificial Intelligence}.}
    \resumeItemListEnd
      
      \resumeProjectHeading
        {\textbf{Working Papers}}{}
        \resumeItemListStart
          \resumeItem{Jingyi Zhang$^\diamond$, Yang Yue, Wei Zhong, \& \textbf{Hao Zeng*} (2025). ``Robust Estimation of Grouped Network Vector Autoregression: An Empirical Analysis Based on China Air Quality Data.'' (Submitted to \textbf{Statistical Research})}
          \resumeItem{\textbf{Hao Zeng}, \& Chao Liu (2025). ``TL-FCSAR: A Transfer Learning Model for Heterogeneous Spatial Effects in U.S. Presidential Elections.'' (Submitted to \textbf{Spatial Statistics})}
          \resumeItem{Wenqian Chen$^\dagger$ , \textbf{Hao Zeng}$^\dagger$ , Jiawen Lei, Yuanning Hou, Chunchun Du, Yiyao Zhang, Yi Lin, Xinan Chen, \& Tianrong He (2025). ``Sex-specific Associations between Craniofacial Skeletal Patterns and Condylar Morphology and Position in Temporomandibular Joint Osteoarthritis: A Three-dimensional Latent Variable Analysis.'' (Submitted to \textbf{Journal of Dentistry})}
          \resumeItem{Huipeng Huang$^\diamond$, Wenbo Liao, Huajun Xi, \textbf{Hao Zeng}, Mengchen Zhao, \& Hongxin Wei (2025). ``Selective Labeling with False Discovery Rate Control.'' (Submitted to \textbf{The 14th International Conference on Learning Representations (ICLR 2026)})}
          \resumeItem{Zhenlong Liu$^\diamond$, \textbf{Hao Zeng}, Weiran Huang, \& Hongxin Wei (2025). ``High-Power Training Data Identification with Provable Statistical Guarantees.'' (Submitted to \textbf{The 14th International Conference on Learning Representations (ICLR 2026)})}
          \resumeItem{\textbf{Hao Zeng}$^\dagger$, Jianguo Huang$^\dagger$\(^\diamond\), Bingyi Jing, Hongxin Wei, \& Bo An (2025). ``PAC Reasoning: Controlling the Performance Loss for Efficient Reasoning.'' (Submitted to \textbf{The 14th International Conference on Learning Representations (ICLR 2026)})}
          \resumeItem{Hongfu Gao, Feipeng Zhang, \textbf{Hao Zeng}, Deyu Meng, Bingyi Jing, \& Hongxin Wei (2025). ``Exploring Imbalanced Annotations for Effective In-Context Learning.'' (Submitted to\textbf{ The 14th International Conference on Learning Representations (ICLR 2026)})}
          \resumeItem{Xuanning Zhou$^\diamond$, Zihao Shi$^\diamond$, \textbf{Hao Zeng}, Xiaobo Xia, Bingyi Jing, \& Hongxin Wei (2025). ``Semi-Supervised Conformal Prediction with Unlabeled Nonconformity Score.'' (Submitted to \textbf{The IEEE/CVF Conference on Computer Vision and Pattern Recognition 2026 (CVPR 2026)})}
          \resumeItem{\textbf{Hao Zeng}, Bingyi Jing, \& Hongxin Wei (2025). ``Conditional Tuning in Conformal Prediction.''} 
          \resumeItem{\textbf{Hao Zeng}, Bingyi Jing, \& Hongxin Wei (2025). ``The Double Descent of Conformal Prediction.''} 
          \resumeItem{\textbf{Hao Zeng}, Kangdao Liu, Bingyi Jing, \& Hongxin Wei (2025). ``On Tuning Bias in Conformal Prediction.''} 
          \resumeItem{\textbf{Hao Zeng}, \& Jianguo Huang$^\diamond$ (2025). ``PAC Accelerate: A Statistically Guaranteed Acceleration Method for Large Generative Models.''}
          \resumeItem{\textbf{Hao Zeng}, \& Jianguo Huang$^\diamond$ (2025). ``Group-Conditional PAC Reasoning: Groupwise Efficient Reasoning with Statistical Guarantees.''}
          \resumeItem{\textbf{Hao Zeng} (2025). ``A Note on the Impossibility of Conditional PAC-Efficient Reasoning.''}
          \resumeItem{\textbf{Hao Zeng}, Jiajun Sun$^\diamond$, Bingyi Jing, \& Wei Zhong (2025). ``Test-time Valid Selection of Conformal Sets via Gaussian Stability.''}
          \resumeItem{\textbf{Hao Zeng}, Junxian Liu$^\diamond$, Bingyi Jing, \& Hongxin Wei (2025). ``Efficient Backward Conformal Prediction.''}
          \resumeItem{\textbf{Hao Zeng}, Wei Zhong, Chuang Wan (2025). ``Selective Inference for the Identified Groups in Network Vector Autoregressive Model.''}
          \resumeItem{\textbf{Hao Zeng}, Huipeng Huang$^\diamond$, Jianguo Huang$^\diamond$, Bingyi Jing, \& Hongxin Wei (2025). ``HyPAC:\@ Cost-Efficient LLM--Human Hybrid Annotation with PAC Error Guarantees.''}
          \resumeItem{Xiaoqi Qiu$^\dagger$$^\diamond$, \textbf{Hao Zeng}$^\dagger$, Bingyi Jing, \& Hongxin Wei (2025). ``LLM Provenance Testing via Constrained Model Confidence Sets.''}
        \resumeItemListEnd
      
      \resumeProjectHeading
        {\textbf{Software}}{}
        \resumeItemListStart
          \resumeItem{Chuang Wan, \textbf{Hao Zeng}, Wei Zhong, \& Changliang Zou (2023). ``MTAFT: Data-Driven Estimation for Multi-Threshold Accelerate Failure Time Model.'' \href{https://cran.r-project.org/web/packages/MTAFT/index.html}{https://cran.r-project.org/web/packages/MTAFT/index.html}}
        \resumeItemListEnd
    \resumeSubHeadingListEnd
\fi

%-----------Conference Presentations-----------
\ifzh
  \section{学术报告与会议}
  \resumeSubHeadingListStart
      \resumeSubSubheading
          {{2021年全国数量经济学博士生学术论坛} \textbf{报告}}{2021年12月,厦门}
      \resumeSubSubheading
          {{2021年厦门大学现代统计学研讨会} \textbf{报告}}{2021年12月,厦门}
      \resumeSubSubheading
          {{NSFC计量建模与经济政策研究基础科学中心项目2023博士生年度论坛} \textbf{邀请报告}}{2023年11月,北京}
      \resumeSubSubheading
          {{全国工业统计学教学研究会青年统计学家协会 2024 年年会暨第二届中国青年统计学家论坛} \textbf{海报展示}}{2024年3月,徐州}
      \resumeSubSubheading
          {{首届国科大与厦大经济管理统计优秀博士毕业生论坛} \textbf{邀请报告}}{2024年6月,厦门}
      \resumeSubSubheading
          {{第二届全国统计与数据科学联合会议} \textbf{邀请报告}}{2024年7月,昆明}
      \resumeSubSubheading
          {{第十三届全国概率统计会议} \textbf{报告}}{2024年11月,厦门}
      \resumeSubSubheading
          {{第三届全国统计与数据科学联合会议} \textbf{海报展示}}{2025年7月,杭州}
      \resumeSubSubheading
          {{第42届国际机器学习大会(ICML)} \textbf{海报展示}}{2025年7月,加拿大温哥华}
      \resumeSubSubheading
          {{2025年厦门大学现代统计学研讨会} \textbf{报告}}{2025年12月,厦门}
  \resumeSubHeadingListEnd
\else
  \section{Conference Presentations}
  \resumeSubHeadingListStart
      \resumeSubSubheading
          {Talk in \textbf{The 2021 Forum for Doctoral Students in Quantitative Economics}}{Dec. 2021, Xiamen}
      \resumeSubSubheading
          {Talk in \textbf{Xiamen University 2021 Symposium on Modern Statistics}}{Dec. 2021, Xiamen}
      \resumeSubSubheading
          {Invited Talk in \textbf{2023 PhD Forum on Econometric Modeling and Economic Policy Research}}{Nov. 2023, Beijing}
      \resumeSubSubheading
          {Poster in \textbf{Conference of the Youth Association of the Chinese Association for Industrial Statistics}}{Mar. 2024, Xuzhou}
      \resumeSubSubheading
          {Invited Talk in \textbf{The 1st Outstanding PhD Forum in Economics and Management Statistics}}{Jun. 2024, Xiamen}
      \resumeSubSubheading
          {Invited Talk in \textbf{The 2nd Joint Conference on Statistics and Data Science}}{Jul. 2024, Kunming}
      \resumeSubSubheading
          {Talk in \textbf{The 13th Conference on Probability and Statistics}}{Nov. 2024, Xiamen}
      \resumeSubSubheading
          {Poster in \textbf{The 3rd Joint Conference on Statistics and Data Science}}{Jul. 2025, Hangzhou}
      \resumeSubSubheading
          {Poster in \textbf{Forty-Second International Conference on Machine Learning}}{Jul. 2025, Vancouver, Canada}     
      \resumeSubSubheading
          {Talk in \textbf{Xiamen University 2025 Symposium on Modern Statistics}}{Dec. 2025, Xiamen}
  \resumeSubHeadingListEnd
\fi

%-----------Academic Services-----------
\section{Academic Services}
\begin{itemize}[leftmargin=0.15in, label={}]
   \item \textbf{Journal Reviewer} 
 {International Statistical Review}, {Journal of Multivariate Analysis}, {Spatial Economic Analysis}, and {Quarterly Journal of Economics and Management}.\\
  \item \textbf{Conference Reviewer} 
 {International Conference on Artificial Intelligence and Statistics (AISTATS), AAAI Conference on Artificial Intelligence(AAAI), International Conference on Machine Learning (ICML), International Conference on Learning Representations (ICLR), IEEE/CVF Conference on Computer Vision and Pattern Recognition (CVPR)}
\end{itemize}

%-----------Awards-----------
\ifzh
  \section{荣誉与奖项}
  \resumeSubHeadingListStart
    \resumeSubSubheading
      {厦门大学经济学院2022-2023学年秋季优秀助教}{2024}
    \resumeSubSubheading
      {Meritorious Winner, Interdisciplinary Contest in Modeling, USA}{2016}
    \resumeSubSubheading
      {中国大学生数学竞赛省级三等奖}{2016}
    \resumeSubSubheading
      {中国大学生数学建模竞赛省级二等奖}{2015}
    \resumeSubSubheading
      {山东大学优秀学生一等奖 (前5\%)}{2015}
    \resumeSubSubheading
      {山东大学“三好学生” (前5\%)}{2015}
  \resumeSubHeadingListEnd
\else
  \section{Awards}
  \resumeSubHeadingListStart
    \resumeSubSubheading
      {Excellent Teaching Assistant in Fall 2022-2023, Xiamen University}{2024}
    \resumeSubSubheading
      {Meritorious Winner, Interdisciplinary Contest in Modeling, USA}{2016}
    \resumeSubSubheading
      {3rd Prize at the Provincial Level, Mathematics Competition for Chinese College Students}{2016}
    \resumeSubSubheading
      {2nd Prize at the Provincial Level, China Undergraduate Mathematical Contest in Modeling}{2015}
    \resumeSubSubheading
      {1st Prize for Outstanding Students (Top 5\%), Shandong University}{2015}
    \resumeSubSubheading
      {Merit Student (Top 5\%), Shandong University}{2015} % 三好学生
  \resumeSubHeadingListEnd
\fi


\end{document}

% !TEX TS-program = xelatex

%-------------------------
% 中文简历(基于英文版)
% Author : Hao Zeng
%------------------------

\documentclass[letterpaper,10pt]{article}

\usepackage{latexsym}
\usepackage[empty]{fullpage}
\usepackage{titlesec}
\usepackage{marvosym}
\usepackage[usenames,dvipsnames]{color}
\usepackage{verbatim}
\usepackage{enumitem}
\usepackage[hidelinks]{hyperref}
\usepackage{fancyhdr}
\usepackage{tabularx}
% \usepackage{fontawesome} % 移除避免 XeLaTeX 字体问题
% \input{glyphtounicode} % XeLaTeX 不需要,且可能缺失
\usepackage[UTF8]{ctex}

\pagestyle{fancy}
\fancyhf{} % 清空页眉页脚
\fancyfoot{}
\renewcommand{\headrulewidth}{0pt}
\renewcommand{\footrulewidth}{0pt}

% 页边距调整
\addtolength{\oddsidemargin}{-0.5in}
\addtolength{\evensidemargin}{-0.5in}
\addtolength{\textwidth}{1in}
\addtolength{\topmargin}{-.5in}
\addtolength{\textheight}{1.0in}

\urlstyle{same}

\raggedbottom
\raggedright
\setlength{\tabcolsep}{0in}

% 章节格式
\titleformat{\section}{
  \vspace{-4pt}\scshape\raggedright\large
}{}{0em}{}[\color{black}\titlerule \vspace{-5pt}]

% PDF可机读/ATS可解析(pdfLaTeX 专用;XeLaTeX 不需要)
\ifdefined\pdfgentounicode
  \pdfgentounicode=1
\fi

%-------------------------
% 自定义命令(沿用英文版的结构)
\newcommand{\resumeItem}[1]{
  \item\small{
    {#1 \vspace{-2pt}}
  }
}

\newcommand{\resumeSubheading}[4]{
  \vspace{-2pt}\item
    \begin{tabular*}{0.97\textwidth}[t]{l@{\extracolsep{\fill}}r}
      \textbf{#1} & #2 \\
      \textit{\small#3} & \textit{\small #4} \\
    \end{tabular*}\vspace{-7pt}
}

\newcommand{\resumeSubSubheading}[2]{
    \item
    \begin{tabular*}{0.97\textwidth}{l@{\extracolsep{\fill}}r}
      \textit{\small#1} & \textit{\small #2} \\
    \end{tabular*}\vspace{-7pt}
}

\newcommand{\resumeProjectHeading}[2]{
    \item
    \begin{tabular*}{0.97\textwidth}{l@{\extracolsep{\fill}}r}
      \small#1 & #2 \\
    \end{tabular*}\vspace{-7pt}
}

\newcommand{\resumeSubItem}[1]{\resumeItem{#1}\vspace{-4pt}}

\renewcommand\labelitemii{$\vcenter{\hbox{\tiny$\bullet$}}$}

\newcommand{\resumeSubHeadingListStart}{\begin{itemize}[leftmargin=0.15in, label={}]}
\newcommand{\resumeSubHeadingListEnd}{\end{itemize}}
\newcommand{\resumeItemListStart}{\begin{itemize}}
\newcommand{\resumeItemListEnd}{\end{itemize}\vspace{-5pt}}

%-------------------------------------------
%%%%%%  简历从这里开始  %%%%%%%%%%%%%%%%%%%%%%%%%%%%

\begin{document}

\begin{center}
    \textbf{\Huge \scshape 曾浩}
    \vspace{0.5em}\\
    {\small 博士后,南方科技大学统计与数据科学系}
\end{center}

\begin{center}
    \small 邮箱:\href{mailto:zenghao.acmail@gmail.com}{zenghao.acmail@gmail.com} $|$
    \href{https://scholar.google.com/citations?user=-EiBHeIAAAAJ&hl=en}{Google Scholar} $|$
    \href{https://zenghao-stat.github.io}{个人主页:zenghao-stat.github.io}
\end{center}

%-----------教育经历-----------
\section{教育经历}
  \resumeSubHeadingListStart
    \resumeSubheading
      {邹至庄经济研究院,厦门大学}{2020年9月 -- 2024年6月}
      {统计学博士,导师:钟威, 许杏柏, 刘拓}{}
    
    \resumeSubheading
      {王亚南经济研究院,厦门大学}{2018年9月 -- 2020年6月}
      {数量经济学硕士,导师:钟威}{转博至统计学博士}
    
    \resumeSubheading
      {数学学院,山东大学}{2014年9月 -- 2018年6月}
      {数学学士(彭实戈班:金融数学与金融工程基地班)}{}
  \resumeSubHeadingListEnd


%-----------研究兴趣-----------
\section{研究兴趣}
 \begin{itemize}[leftmargin=0.15in, label={}]
    \small{\item{
     \textbf{统计机器学习}{:机器学习不确定性量化、共型预测、迁移学习} \\
     \textbf{交叉学科研究}{:计量经济学、空间统计、生物统计}
    }}
 \end{itemize}

%-----------经历-----------
\section{工作与教学经历}
  \resumeSubHeadingListStart
    \resumeSubheading
      {南方科技大学、新加坡国立大学联合博士后项目}{2024年7月 -- 至今}
      {统计学博士后,导师:荆炳义,魏鸿鑫,周望}{}
    
    \resumeSubheading
      {统计与数据科学系, 新加坡国立大学}{2023年5月 -- 2023年10月}
      {访问学生,导师:余涛}{}
    \resumeSubheading
      {经济学院,厦门大学}{2019年 秋季 -- 2022年 秋季}
      {助教:高级计量经济学 I(研)、高级概率论(研)、实变分析(本)两次、概率论导论(本)}{}
  \resumeSubHeadingListEnd


%-----------论文与出版-----------
\section{学术成果}
(* 表示通讯作者,$^\dagger$ 表示共同贡献。)
  \resumeSubHeadingListStart
  \resumeProjectHeading{\textbf{期刊文章}}{}
      \resumeItemListStart
        % \resumeItem{\textbf{2025}}
        \resumeItem{Liu, K., Sun, T., \textbf{Zeng, H.}, Zhang, Y., Pun, C.-M., \& Vong, C.-M. (2025). ``Spatial-Aware Conformal Prediction for Trustworthy Hyperspectral Image Classification.'' \textit{\textbf{IEEE Transactions on Circuits and Systems for Video Technology}}. (中科院一区,计算机最优期刊)
        }
        % \resumeItem{\textbf{2024}}
        \resumeItem{\textbf{Zeng, H.}, Wan, C., Zhong, W., \& Liu, T. (2024). ``Robust Integrative Analysis via Quantile Regression with Homogeneity and Sparsity.'' \textit{\textbf{Journal of Statistical Planning and Inference}}.
        }
        \resumeItem{Wan, C.$^\dagger$, \textbf{Zeng, H.*$^\dagger$}, Zhang, W.$^\dagger$, Zhong, W.$^\dagger$, \& Zou, C.$^\dagger$. (2024). ``Data‐driven Estimation for Multithreshold Accelerated Failure Time Model.'' \textit{\textbf{Scandinavian Journal of Statistics}}.}
      \resumeItemListEnd
      
    \resumeProjectHeading{\textbf{会议文章}}{}
    \resumeItemListStart
        \resumeItem{\textbf{Zeng, H.$^\dagger$}, Liu, K.$^\dagger$, Jing, B., \& Wei, H. (2025). ``Parametric Scaling Law of Tuning Bias in Conformal Prediction.'' \textit{\textbf{42nd International Conference on Machine Learning}}. (人工智能顶级会议)}
        \resumeItem{Xi, H., Liu, K., \textbf{Zeng, H.}, Sun, W., \& Wei, H. (2025).``Robust Online Conformal Prediction under Uniform Label Noise.'' \textit{\textbf{39th Annual Conference on Neural Information Processing Systems}}. (人工智能顶级会议)
        } 
        \resumeItem{Liu, K., \textbf{Zeng, H.}, Huang, J., Zhuang, H., Vong, C.-M., \& Wei, H. (2025). ``C-Adapter: Adapting Deep Classifiers for Efficient Conformal Prediction Sets.'' \textit{\textbf{28th European Conference on Artificial Intelligence}}.}
    \resumeItemListEnd
    
    \resumeProjectHeading
      {\textbf{工作论文}}{}
      \resumeItemListStart
        \resumeItem{\textbf{Zeng, H.}, Jing, B., \& Wei, H. (2025). ``Conditional Tuning in Conformal Prediction.''}
        \resumeItem{\textbf{Zeng, H.}, Jing, B., \& Wei, H. (2025). ``The Double Descent of Conformal Prediction.''}
        \resumeItem{\textbf{Zeng, H.}, Liu, K., Jing, B., \& Wei, H. (2025). ``On Tuning Bias in Conformal Prediction.''}
        \resumeItem{Huang, H., Liao, W., Xi, H., \textbf{Zeng, H.}, Zhao, M., \& Wei, H. (2025). ``Selective Labeling with False Discovery Rate Control.'' (Submit to ICLR 2026)}
        \resumeItem{Liu, Z., \textbf{Zeng, H.}, Huang, W., \& Wei, H. (2025). ``High-Power Training Data Identification with Provable Statistical Guarantees.''(Submit to ICLR 2026)}
        \resumeItem{\textbf{Zeng, H.}$^\dagger$, Huang, J.$^\dagger$, Jing, B., Wei, H., \& An, B. (2025). ``PAC Reasoning: Controlling the Performance Loss for Efficient Reasoning.''(Submit to ICLR 2026)}
        \resumeItem{Gao, H., Zhang, F., \textbf{Zeng, H.}, Meng, D., Jing, B., \& Wei, H. (2025). ``Exploring Imbalanced Annotations for Effective In-Context Learning.''(Submit to ICLR 2026)}
        \resumeItem{\textbf{Zeng, H.}, Zhong, W., \& Xu, X. (2024). ``Transfer Learning for Spatial Autoregressive Models with Application to U.S. Presidential Election Prediction.'' (返修至\textit{\textbf{Journal of Business \& Economic Statistics}})} 
      \resumeItemListEnd
    
    \resumeProjectHeading
      {\textbf{开源软件}}{}
      \resumeItemListStart
        \resumeItem{Wan, C., \textbf{Zeng, H.}, Zhong, W., \& Zou, C. (2023). ``MTAFT: Data-Driven Estimation for Multi-Threshold Accelerate Failure Time Model.'' \href{https://cran.r-project.org/web/packages/MTAFT/index.html}{https://cran.r-project.org/web/packages/MTAFT/index.html}}
      \resumeItemListEnd
  \resumeSubHeadingListEnd


  \section{荣誉与奖项}
  \resumeSubHeadingListStart
    \resumeSubSubheading
      {厦门大学经济学院2022-2023学年秋季优秀助教}{2024}
    \resumeSubSubheading
      {Meritorious Winner, Interdisciplinary Contest in Modeling, USA}{2016}
    \resumeSubSubheading
      {中国大学生数学竞赛省级三等奖}{2016}
    \resumeSubSubheading
      {中国大学生数学建模竞赛省级二等奖}{2015}
    \resumeSubSubheading
      {山东大学优秀学生一等奖}{2015}
    \resumeSubSubheading
      {山东大学“三好学生”}{2015}
  \resumeSubHeadingListEnd


\section{学术报告与会议}
\resumeSubHeadingListStart
    \resumeSubSubheading
        {{2021年全国数量经济学博士生学术论坛} \textbf{报告}}{2021年12月,厦门}
    \resumeSubSubheading
        {{2021年厦门大学现代统计学研讨会} \textbf{邀请报告}}{2021年12月,厦门}
    \resumeSubSubheading
        {{NSFC“计量建模与经济政策研究”基础科学中心项目2023博士生年度论坛} \textbf{邀请报告}}{2023年11月,北京}
    \resumeSubSubheading
        {{全国工业统计学教学研究会青年统计学家协会 2024 年年会暨第二届中国青年统计学家论坛} \textbf{海报展示}}{2024年3月,徐州}
    \resumeSubSubheading
        {{首届国科大与厦大经济管理统计优秀博士毕业生论坛} \textbf{邀请报告}}{2024年6月,厦门}
    \resumeSubSubheading
        {{第二届全国统计与数据科学联合会议} \textbf{邀请报告}}{2024年7月,昆明}
    \resumeSubSubheading
        {{第十三届全国概率统计会议} \textbf{作报告}}{2024年11月,厦门}
    \resumeSubSubheading
        {{第三届全国统计与数据科学联合会议} \textbf{海报展示}}{2025年7月,杭州}
    \resumeSubSubheading
        {{第42届国际机器学习大会(ICML)} \textbf{海报展示}}{2025年7月,加拿大温哥华}
\resumeSubHeadingListEnd


  \section{学术服务}
  期刊审稿人:\textit{International Statistical Review},\textit{Journal of Multivariate Analysis},\textit{Spatial Economic Analysis},\textit{Quarterly Journal of Economics and Management}。\\
  会议审稿人:\textit{International Conference on Artificial Intelligence and Statistics (AISTATS)},\textit{Association for the Advancement of Artificial Intelligence (AAAI)},\textit{International Conference on Machine Learning (ICML)},\textit{International Conference on Learning Representations (ICLR)}

\end{document}